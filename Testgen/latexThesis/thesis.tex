\documentclass[12pt,oneside]{book}
\usepackage{fullpage}
\usepackage{color}
\usepackage{mathtools}
\usepackage{amsmath}
\usepackage{amsfonts}
\usepackage{graphicx}
\usepackage{lipsum}
\usepackage[utf8]{inputenc}
\usepackage{geometry}
\usepackage{fancyhdr}
 

\rfoot{Page \thepage}

\geometry{
 a4paper,
 total={210mm,297mm},
 left=35mm,
 right=25mm,
 top=25mm,
 bottom=25mm,
}
\linespread {1.6}

\newif\ifquoteopen
\catcode`\"=\active % lets you define `"` as a macro
\DeclareRobustCommand*{"}{%
   \ifquoteopen
     \quoteopenfalse ''%
   \else
     \quoteopentrue ``%
   \fi
}

%Gummi|065|=)

\date{}
\begin{document}

%% ----------------------------------------------PAGE 1 TITLE

\begin{titlepage}
    \begin{center}
        \vspace*{1cm}
        
        {\scshape {\huge{I}\Large{NTER-PROCEDURAL} \huge{A}\Large{NALYSIS OF}\\
                   \vspace{0.1cm}
				   \huge{C}\Large{ONCOLIC } \huge{E}\Large{XECUTION}}}\\[1.5cm] % Thesis title

        
        \textit{A Thesis Submitted\\
        in Partial Fulfilment of the Requirements\\
        for the Degree of\\}
                \vspace{0.8cm}
        \textbf{\large{MASTER OF TECHNOLOGY}}\\[0.5\baselineskip]
        \textit{by}\\[0.5\baselineskip]
        \textbf{\large{Ashwini Kshitij}}\\
                \vspace{0.8cm}
        \textit{supervised by}\\
                \vspace{0.8cm}
        \textbf{\large{Dr. Subhajit Roy}}\\
        \textbf{\large{Dr. Amey Karkare}}
        \vspace{0.8cm}
        
      	\includegraphics[width=0.25\textwidth]{iitklogo.png}
        
        \vspace{1\baselineskip}
        
        \uppercase{Department of Computer Science and Engineering\\
        \textbf{Indian Institute of Technology Kanpur}}\\
        
        \vspace{1\baselineskip}
        \textbf{June 2015}
        
    \end{center}
\end{titlepage}


%%-----------------------------------------------PAGE 2 Certificate

\frontmatter
\addcontentsline{toc}{chapter}{Certificate}
\begin{center}
\textbf{\Large{CERTIFICATE}}
\end{center}
\vspace*{3\baselineskip}
This is to certify that the work contained in this thesis entitled \textbf{\textit{"Inter-procedural analysis of Concolic Execution"}}, by \textbf{Ashwini Kshitij (Roll No. 10327165)}, has been carried out under my supervision and this work has not been submitted elsewhere for a degree.\\
\\[4\baselineskip]

  \small
  \vspace{2.0 cm}
  \begin{tabular*}{1.0\textwidth}{@{\extracolsep{\fill}} l r}
    \textbf{Dr. Subhajit Roy} 			 & 				\textbf{Dr. Amey Karkare}\\
    Assistant Professor,				 &				Assistant Professor,\\
    Department of CSE, 					 & 				Department of CSE,\\
    IIT Kanpur.							 &				IIT Kanpur.
    
  \end{tabular*}
\newpage


%%----------------------------------------------ABSTRACT
\addcontentsline{toc}{chapter}{Abstract}
\begin{center}
\textbf{\Large{ABSTRACT}}
\end{center}
\vspace {0.8 cm}

Interprocedural analysis is the cornerstone of determining precise program behavioral information. Using this technique we can avoid making overly conversative assumptions about the effects of procedures and the state at call sites. It aims at gathering informations across multiple procedures.  

\vspace {0.5 cm}

In this thesis, to extend the concept of concolic execution to interprocedural calls we instrument the function call site. For such instrumentation purposes we have used a tool called CIL. Just before the function call we set up the environment for calling the procedure which in turn enables the it follow through the symbolic execution of parameters with the control flow from caller to callee. Similarly, just after the call instruction we restore the calling environment by restoring the symbolic values.

\vspace {0.5 cm}

Modifying the symbolic execution engine to collect the interprocedural analysis information can have widepread applications in software verification and testing. We can improve the coverage of test suites that are automatically generated by intraprocedural concolic executers.   

  






\newpage

%%----------------------------------------------ACKNOWLEDGMENT
\addcontentsline{toc}{chapter}{Acknowledgement}
\begin{center}
\textbf{\Large{Acknowledgements}}
\end{center}

\vspace{1.0 cm}

I acknowledge, with gratitude, my debt of thanks to Professor Subhajit Roy for his advise and encouragement and to Professor Amey Karkare for his aid and foresight. They presented me with the opportunity to tackle interesting problems in field of Software Testing. Their patient but firm guidance was critical to successful completion of my research.  

\vspace{0.5 cm}

I appreciate the support of my friends and wingmates who always provided me with the confidence and courage to tackle even the most challenging problems. Without their help and councel, the completion this work would have been immeasurably more difficult. 

\vspace{1.0 cm}

I also want to express my sincere gratitude to \textit {\textbf {BRNS}} for encouraging our research work. Their unshaken faith in our efforts motivated me to follow through this project.

\begin{flushright}
\textbf{-Ashwini Kshitij}
\end{flushright}

\newpage

%%----------------------------------------------CONTENTS

\tableofcontents
\mainmatter

%%---------------------------------------------CHAP 1 Intro

\chapter{Introduction}
%%\lhead{Chapter 1. \emph{Introduction}}

\hspace {0.4 cm}
Symbolic Execution has been a recipient of significant attention during the past few years. It is now considered an effective technique in generation of high coverage test suites. The idea has been discovered around three decades ago but it was after significant improvements the potential of idea is harnessed. One such important improvement was symbolic execution alongside keeping track of concrete values (concolic executic). The main advantage of this technique is that whenever constraint solving complications (like timeouts) occur during classical symbolic execution, it is alleviated using the concrete values.

More practical programs involve procedure calls. The symbolic execution goes smoothly if there are no function calls involve. A function call tranfers the control from the caller to the callee. That function may very well modify the symbolic state of the variables. If we transition the symbolic state correctly through the function call, the symbolic execution will run as usual with the modified symbolic state.   


\section{Problem Statement}
The implementation of this thesis is built into a tool 'Testgen' that does automatic test case generation. The tool uses inbuilt concolic execution engine for the purpose. But one of the drawbacks the tool faces is whenever a procedure call is encountered, the symbolic value is reset to "start state" instead of modifying it as per the operations performed on it by the callee function.  

More details than abstract


\section {Contributions}



\section {Organisation of thesis}




\newpage
\chapter{Background}

\section {Concolic Testing}

\section {Intraprocedural Analysis}

\end{document}
